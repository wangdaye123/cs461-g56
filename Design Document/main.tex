\documentclass[10pt,draftclsnofoot,onecolumn,journal,compsoc]{IEEEtran}

\usepackage[margin=0.75in]{geometry}
\usepackage{graphicx}
\usepackage{caption}
\usepackage{hyperref}
\usepackage{enumerate}
\usepackage{tabu}
\usepackage[english]{babel}\usepackage[numbers]{natbib}
\usepackage{natbib}
\usepackage{longtable}
\usepackage{listings}
\usepackage{pgfgantt}
\usepackage{vhistory}

\renewcommand{\linespread}{1.0}
\renewcommand{\bibsection}{}

\lstset{
captionpos=b,           
}

\renewcommand{\linespread}{1.0}
\title{Prototype Big Data Archive in a Public Cloud}
\author{
  \IEEEauthorblockN{Group 56: Pathfinder of Big Data\\Zhi Jiang, Isaac T Chan, Zhaohensg Wang} \\
  \IEEEauthorblockA{CS 461: Senior Capstone Fall 2016 \\ Oregon State University}
}
\date{}

\IEEEtitleabstractindextext{
	\begin{abstract}
OSU campuses generate data constantly from multiples sources, including computer labs, wireless usage, student devices, and many others. This quantity of data, also known as big data which will effectively represent all kinds of behaviors of students for information technology. However, the data is very difficult to manage because it is collected from multiple sources and is hard to analyze. As a result, various technologies will be required for supporting this project. In this document, we will talk about these technologies based on different viewpoints such as context, information or algorithm. For each viewpoint, we will declare the design concerns for it. Besides, we will introduce our technology detail based on these viewpoints.
	\end{abstract}
}


\begin{document}
% cover page    
    \maketitle
    \IEEEdisplaynontitleabstractindextext
    \IEEEpeerreviewmaketitle
    \newpage
% catalog    
    \tableofcontents
    \newpage

% revision history
	\begin{versionhistory}
  \vhEntry{1.0}{2017.02.06}{Zhi Jiang}{created revision history}
  \vhEntry{1.1}{2017.02.07}{Zhi Jiang}{modified the time line}
  \vhEntry{1.2}{2017.02.10}{zhaoheng wang}{modified the design part}
\end{versionhistory}
	\newpage

% content   
\section{Introduction}
	\subsection{Purpose}
The purpose of this document is to elaborate implementation of main technologies used in this product. We will introduce each technology from several design viewpoints such as interaction and structure. The intended audiences of this document include the client, development group, and assessors of this product. The client can understand all details of compositions that the developer will implement in this product according to this document. On the other hand, the development group can implement special plans such testing plan based on attributes of these technologies. Assessors can evaluate technologies used in the final product and then compare them with this document to ensure final product matches these technologies.  

\subsection{Scope}
In this document, technologies are separated into eight distinct pieces and each member will discuss their own pieces individually. Isaac Chan is in charge of these pieces: methods to measure performance metrics of database functionality, methods of database security, and methods of user interaction with the system. Zhi Jiang is in charge of these three pieces: S3, Amazon Kinesis, and Data Pipeline. Zhaoheng Wang is in charge of DynamoDB table design, DynamoDB Functionality implementation, and Quicksight Visualization.

\subsection{Summary}
In the document, section one will give a brief introduction for the whole document. Section two and three declare reference terms for the document. From section four, we start introducing our own technologies based on different viewpoints context, information or algorithm. For each viewpoint,we will declare the design concerns for it. This document will be in much more detail than the technology review document.  
 


\section{References}
	\section{References}
\begin{thebibliography}{10}
   	\bibitem{g1}"Elastic Compute Cloud (EC2) Cloud Server and Hosting – AWS", \textit{Amazon Web Services, Inc.}, 2016. [Online]. Available: https://aws.amazon.com/ec2/?nc1=h\_ls. [Accessed: 30- Nov- 2016].
    \bibitem{z1}\textit{Amazon Simple Storage Service - Developer guide}, 1st ed. Amazon Web Services, Inc, 2016, p. 3.
  	\bibitem{z2}\textit{Amazon Simple Storage Service - Developer guide}, 1st ed. Amazon Web Services, Inc, 2016, p. 196.
    \bibitem{z3}\textit{Amazon Kinesis Firehose - Developer guide}, 1st ed. Amazon Web Services, Inc, 2016, p. 2.
  	\bibitem{z4} Boto3.readthedocs.io. (2014). \textit{Boto 3 Documentation — Boto 3 Docs 1.4.4 documentation}. [online] Available at: http://boto3.readthedocs.io/en/latest/index.html [Accessed 16 Feb. 2017].
  	\bibitem{z6}\textit{AWS Data Pipeline - Developer Guide}, 1st ed. Amazon Web Services, Inc, 2016, p. 148.
  	\bibitem{z7}\textit{AWS Data Pipeline - Developer Guide}, 1st ed. Amazon Web Services, Inc, 2016, p. 136.
  
  	\bibitem{w1}"Amazon DynamoDB Developer Guide", \textit{Amazon Web Services, Inc.}, 2016. [Online]. Available: 	      http://docs.aws.amazon.com/amazondynamodb/latest/developerguide/JavaDocumentAPIWorkingWithTables.html. [Accessed: 1- DEC- 2016].
  	\bibitem{w2}"Amazon DynamoDB Developer API for Querying Tables and Indexes", \textit{Amazon Web Services, Inc.}, 2016. [Online]. Available:
  http://docs.aws.amazon.com/amazondynamodb/latest/developerguide/QueryingJavaDocumentAPI.html. [Accessed: 1- DEC- 2016].
  	\bibitem{w3}"Amazon DynamoDB Developer API for Scanning Tables and Indexes", \textit{Amazon Web Services, Inc.}, 2016. [Online]. Available:
  http://docs.aws.amazon.com/amazondynamodb/latest/developerguide/ScanJavaDocumentAPI.html. [Accessed: 1- DEC- 2016].
  	\bibitem{w4}"Amazon QuickSight User Guide", \textit{Amazon Web Services, Inc.}, 2016. [Online]. Available:
  http://docs.aws.amazon.com/quicksight/latest/user/getting-started-create-analysis-database.html. [Accessed: 1- DEC- 2016].
  
	\bibitem{i1}"Amazon Cloudwatch PutMetricData API", \textit{Amazon Web Services, Inc.}, 2016. [Online]. Available:
    
http://docs.aws.amazon.com/AmazonCloudWatch/latest/APIReference/API\_PutMetricData.html. [Accessed: 1- DEC- 2016]

	\bibitem{i2}"How to collect RDS MySQL metrics", \textit{John Matson}, 2016. [Online]. Available:
    
https://www.datadoghq.com/blog/how-to-collect-rds-mysql-metrics/. [Accessed: 1- DEC- 2016]
    
    \bibitem{i3}"Identity and Access Management (IAM)", \textit{Amazon Web Services, Inc}, 2016. [Online]. Available:
    
https://aws.amazon.com/iam/. [Accessed: 1- DEC- 2016]
  
 	\bibitem{i4}"Amazon EMR", \textit{Amazon Web Services, Inc}, 2016. [Online]. Available:
    
https://aws.amazon.com/emr/?nc2=h\_l3\_al. [Accessed: 1- DEC- 2016]

	\bibitem{i5}"Simple Linear Regression with Pure Python", \textit{ActiveState Code}, 2014. [Online]. Available:
    
http://code.activestate.com/recipes/578914-simple-linear-regression-with-pure-python/. [Accessed: 1- DEC- 2016]

\end{thebibliography}

\section{Glossary}
	\begin{tabu} to \hsize {|X|X[2,l]|}
        \hline
        \textbf{Term} & \textbf{Definition}\\
        \hline
        User & People who interact with our project \\
        \hline
        DB & Database\\
        \hline
        Nosql & Non-relational database\\
        \hline
        SDK & Software development kit\\
         \hline
        Schema & Database table \\
        \hline
        AWS & Amazon web service, a Platform as a service(PaaS) offered by Amazon \\
        \hline
        S3 & Simple storage service provided by Amazon\\
        \hline
        Amazon Kinesis & A service used to process stream and log file data \\
        \hline
        SQL & a standard programming language used to access and process database or data storage\\
        \hline
        EC2 & a web service that provides resizable compute capacity in the cloud\cite{g1}\\
        \hline
        EMR & A comprehensive tool that can manage data and provide analysis through conventional queries or machine learning technology\\
		\hline
\end{tabu}

\section{Timeline}
	\begin{ganttchart}[vgrid, hgrid]{1}{18}
        \gantttitle{Winter}{10}
        \gantttitle{Spring}{8}\\
        
        \gantttitlelist{1,...,10}{1}
        \gantttitlelist{1,...,8}{1}\\
        
        \ganttbar{Data storage implementation}{1}{1}\\ 
        \ganttbar{Database table implementation}{1}{2}\\
        \ganttbar{Application of parsing data}{2}{4}\\
        \ganttbar{Application of transforming data}{4}{6}\\
        \ganttbar{Data visualization}{7}{8} \\
        
        \ganttbar{Test for functionality}{9}{11} \\
        \ganttbar{Performance optimization}{10}{13}\\
        \ganttbar{Security optimization}{13}{16}\\
        \ganttbar{Cost comparison}{17}{18}
 \end{ganttchart}
    
%body
\section{AWS Cloudwatch}
    \subsection{Context}
    Performance metrics for database functionality is an important element of the implementation. We will assess performance from typical database operations: data inserts, updates, and reads. After reviewing different technologies, AWS Cloudwatch is the utility we will use to perform these operations and measure the performance.
    
    \subsection{Viewpoint: Users}
    We will have one primary user for the implemented system: OSU staff who perform data analysis. From their viewpoint, performance is an incredibly important feature of the database. The data analyzer may be performing analysis on potentially extremely large data sets; if the speed of reading the data in the database is slow, it is not a successful implementation. We understand that “slow” is a vague term, but without a method of comparison it is impossible to judge the collected performance metrics as of now. We will meet with current data analyzers and our client to determine whether or not the performance meets expectations.\\
    
    \noindent In order to accurately provide a model for realistic usage, we will need to measure performance of database operations using differing sizes of sample data.
    
    \subsection{Viewpoint: Scalability}
    Another large aspect of our database implementation is the ability for it to scale well in the future, with more and more data added. Although this will of course affect the user, scalability will also affect the database administrator. Performance for inserting and updating data within the database is directly related to scalability, and whether or not the performance is heavily affected by the amount of stored data.\\
    
    \noindent We will use large sets of sample data to load into the database. These large sizes should again, differ in size in order to model an estimation for future database performance.
    
    \subsection{Implementation}
    In AWS Cloudwatch, we will utilize the built-in PutMetricData API to input custom metrics for Cloudwatch to monitor\cite{i1}. This API works by passing measurements of interest into the API and the measurements are saved to a location within AWS for us to view.\\
    
	\noindent Here is a code example of checking CPU utilization, using the Cloudwatch command line interface. As shown, it displays the namespace, as well as the runtime. After collecting these types of metrics using differing sizes of data, we can provide an accurate model and estimate projected metrics with extremely large data sets.

\begin{lstlisting}[caption=Cloudwatch monitoring example\cite{i2}]
mon-get-stats CPUUtilization   
      --namespace="AWS/RDS"     
      --dimensions="DBInstanceIdentifier=instance-name"
      --statistics Maximum
      --start-time 2015-09-29T00:00:00   
      --end-time 2015-09-29T00:05:00
\end{lstlisting}

\section{Database Security}
	\subsection{Context}
    With methods database security, we hope to improve the security of our database by restricting access and preventing malicious utilization of data. Our best options for database security is AWS’s user authentication policies, encrypting sensitive data fields, and using sufficient input validation to avoid injection attacks.
    \subsection{Viewpoint: Users}
    From a data analyzer viewpoint, database security is a concern, in order to prevent unintended data loss, either from malicious users or unfortunate accidental injection attacks. Data loss, if unrecognized while performing analysis, can result in incorrect conclusions drawn from incomplete data. Additionally, waiting for backups to be restored or missing data to be re-inserted takes up time and hinders work.
    
    \subsection{Adminstrators}
    From an administration perspective, database security is a concern to reduce the possibility of malicious user access. Administrators will carefully restrict the number of authenticated users using AWS’s user authentication policies. Most of the data inside the database is OSU’s system users, with identifying fields such as student ID’s. If a malicious user were to gain access to the database, it would be a massive privacy breach for them to have access to identifying features of such high volume. Administrators will encrypt these identifying fields to prevent identification of the data.\\
    
    \noindent Elements of course include the database and subsequent data. Additionally, after we assess the necessity of including backups, possibly backups. 
    
    \subsection{Implementation}
    AWS’s user authentications, also known as AWS IAM (Identity and Access Management), is quite robust. Administrators can create and manage AWS users and groups, while setting permissions to restrict access to certain AWS resources\cite{i3}. For example, administrators can set modularize groups to only have access to data ingestion, processing, or database for analysis. This prevents a large number of users to have access to the whole system and increases liability and responsibility for the users.\\
    
	\noindent As we implement the database, the user-identifying fields within the data is already encrypted to protect us as developers and reduce liability for our client. This encryption will remain within the database and future data inserts will be encrypted as according to the administrators.\\
    
	\noindent Finally, minimizing possibility of injection attacks. There is inherent protection when user authentication is done properly, because authenticated users will not be as likely to perform malicious operations. However, we will be implementing our NoSQL database with AWS DynamoDB, which inherently prevents injection attacks by not allowing multiple operations within one command.
    
\section{User Interaction}
	\subsection{Context}
    Users will need to be able to interact with the system. They will need to use different utilities for different interactions, such as monitoring resources, monitoring performance, managing data, and performing different methods of analysis. As discussed previously regarding performance metrics, the utility of choice for monitoring is AWS Cloudwatch. For the different methods of analysis, we chose AWS EMR as the most well-rounded utility for comprehensive coverage of analysis techniques.
    
    \subsection{Viewpoint: Data analyzers}
    From the viewpoint of a data analyzer, they will be of course interested mainly in performing analysis on the data within the database. It is their choice what utility they use, but as the developers we would like to ensure the database implementation is acceptable. We chose AWS EMR as the utility we will test our database with. From EMR, we can perform basic tests to ensure data analysis is possible.\\ 
    
    \noindent AWS EMR is a managed Hadoop framework in a command line interface\cite{i4}. Scripts and code can be run from AWS EMR. As developers, we will be running basic data analysis code from EMR against the implemented database to ensure analysis is possible.
    
    \subsection{Viewpoint: Administrators}
    From an administrative perspective, they will be interacting with the system with monitoring and management utilities. Because monitoring was covered in-depth in the performance metrics section, this section will mainly be on the management utility. Database management includes operations such as inserting new or updating data within the database. This can be achieved using our database implementation utility, AWS DynamoDB. See section 10, DynamoDB for more details.
    
    \subsection{Implementation}
    Provided is a sample python script to perform a linear regression analysis on a set of data. Obviously as of now, the analysis is purely hypothetical, considering we do not currently have a source of data, nor we do know how the data will be analyzed. This sample code is only a proof of concept that scripts such as these can be run within AWS EMR to ensure analysis is possible with our database implementation.
    
\begin{lstlisting}[caption=Sample EMR linear regression analysis example\cite{i5}]
def fit(X, Y):

    def mean(Xs):
        return sum(Xs) / len(Xs)
        
    m_X = mean(X)
    m_Y = mean(Y)

    def std(Xs, m):
        normalizer = len(Xs) - 1
        return math.sqrt(sum((pow(x - m, 2) for x in Xs)) / normalizer)

    def pearson_r(Xs, Ys):
        sum_xy = 0
        sum_sq_v_x = 0
        sum_sq_v_y = 0

        for (x, y) in zip(Xs, Ys):
            var_x = x - m_X
            var_y = y - m_Y
            sum_xy += var_x * var_y
            sum_sq_v_x += pow(var_x, 2)
            sum_sq_v_y += pow(var_y, 2)
        return sum_xy / math.sqrt(sum_sq_v_x * sum_sq_v_y)

    r = pearson_r(X, Y)
    b = r * (std(Y, m_Y) / std(X, m_X))
    A = m_Y - b * m_X

    def line(x):
        return b * x + A
    return line
\end{lstlisting}
\section{S3}
	\subsection{Context}
	The purpose of S3, as a data storage, is used to store all files, and these files include unformatted and uncleaned data and results integrated into analytics tool. S3 gains any types of data and format data from the user and then move them to other services in Amazon platform.
        
	\subsection{Composition}
	\textbf{Buckets:} bucket is a basic container used to store objects in S3.\cite{z1}
     
    \noindent \textbf{Objects:} Objects are the fundamental entities stored in S3, and it consists of object data and metadata. The role of metadata is to store set of name-value pairs that describe the object.\cite{z1}
        
	\noindent \textbf{Keys:} each object has the unique identifier and which is called “key”. In order to ensure uniqueness of each object, the combination of a bucket, key and version ID is used to identify each object in S3.\cite{z1}
    
	\subsection{Interaction}
    S3 need to upload data to analytics and send data to database, thus S3 will interact with the Data Pipeline, and then complete data transforming between different compute and storage services according to the application on data pipeline. On the other hand, the developer needs to program some functions for S3, thus S3 will interact with AWS Lambda. The role of AWS Lambda is to run code for all backed services on this platform.
    
	\subsection{Algorithm}
    The AWS SDK supports several programming languages to develop S3 by programming. The programming languages cover Java, .NET, Ruby, Python and PHP. On the other AWS SDK also provide many API for these programming languages. Take an instance with Java, the developer can utilize low-level API to implement create, update, and delete operations that apply to buckets and objects in S3.\cite{z2} 
   
\section{Amazon Kinesis}
	\subsection{Context}
    Parsing data like log file and streaming can ensure consistency and normalization of data which will be moved to database. Thus, the service Amazon Kinesis provides is complete the basic integrating and parsing for these data.
        
	\subsection{Composition}
    \textbf{Amazon Kinesis Firehose:} the key concept of Amazon Kinesis Firehose is to create a delivery stream and then the delivery stream is used to receive data from the user. Eventually, this delivery stream will be processed by Amazon Kinesis Analytics as the underlying entity of Firehose.\cite{z3}
    
    \noindent \textbf{Amazon Kinesis Analytics:} Amazon Kinesis Analytics support user uses a series of SQL statements to complete multiple operations for the stream. On the other hand, the developer can also develop own application to satisfy some special requirements.
     
    \noindent \textbf{Amazon Kinesis Streams:} the developer can create a stream through this composition.

	\subsection{Dependency}
    The service is not completely independent from others. The reason is it needs data from the external source such as S3, so the precondition of using this service is developer must have existing data.

	\subsection{Interaction}
    Amazon Kinesis must interact with other entities. There are two reasons. One is Amazon Kinesis needs to receive data from other services. Second reason is the results should be exported to other services after data parsed by Amazon Kinesis Analytics. 

	\subsection{Algorithm}
    The core concept of Amazon Kinesis Analytics is to create one or more applications for completing multiple tasks. The programming language is SQL in this composition, and it provides enough functions to solve some general problems.\\

	\noindent \textbf{Pre-processing streams:} The purpose of pre-processing streams is to normalize data and avoid errors appear before some applications do high-level processing because sometimes streaming source contains many kinds of extra conditions. For example, if the streaming source includes multiple record types, which means developers need to integrate all data from these two record types. In this condition, developers can create two additional in-application streams to store these two record types separately. And the filter the rows from the original source based on record type and insert them in the newly created streams using pumps.\cite{z4}\\

	\noindent \textbf{Most frequently occurring values:} seeking most frequently is also a usual operation for data analysis. For example, according to our client’s requirements, developers need to find which printer are most commonly used by students, so in the other word, the application must be able to calculate the frequently occurring values of “printer” column in the table. The function \textbf{TOP\_K\_ITEMS\_TUMBLING} provided by AWS can effectively deal with this kind of problem. Developers still can set some extra conditions or constraints such as finding the top three most frequently printer in the library.\\

	\noindent \textbf{Date anomalies:} client needs to ensure everything is normal, so developers need to detect data anomalies on a stream in order resolve any potential issues. The AWS also provide a function to complete this kind of task, and which is \textbf{RANDOM\_CUT\_FOREST}. Take an instance, if the client needs to find a condition of delay time for the certain webpage, the developer can assign an anomaly score in this function, and then output results including all of the anomaly delay time.
       
\section{Data Pipeline}
	\subsection{Context}
    The product relates to a number of different compute and storage services thus function of AWS Data Pipeline is to help user efficiently and massively move data among these services.
     
     \subsection{Composition}
     \textbf{Data node:} entity of data source in the pipeline, and attributes of it consist of name, locations, and formats\cite{z5}. 
     
     \noindent\textbf{Activity:} activity represents methods to transform data such as moving data from one location to another. 
     
     \noindent\textbf{Schedule:} each activity has own schedule for operating data
     
     \noindent\textbf{Resources:} entity that implements activities when they are scheduled 
     
	\subsection{Dependency}
     Data Pipeline must depend on other services because the main goal of it is to transform data from other data sources. On the other hand, the activities are extensible, so the developer is allowed to run custom scripts to implement more combinations.
     
    	\subsection{Structure}
The following diagram represents a complete structure of the Data Pipeline. 
    \begin{figure}[h]
        \includegraphics[width=10cm, height=9cm]{data_pipeline.png}
        \centering
        \caption{Complete structure of Data Pipeline}
    \end{figure}
    
	\subsection{Interaction}
	AWS provides management console to create a data pipeline directly. The developer can set the data source and data destination, and then the new data pipeline will automatically make  a connection between these two locations.  
    
\subsection{ Algorithm}
	In this product, the locations are S3 and DynamoDB for data pipeline, thus all of algorithms must be associated to these two services. AWS Data Pipeline supports \textbf{S3DataNode} and \textbf{DynamoDBDataNode}. The  object example of \textbf{S3dataNode} is
	\begin{lstlisting}[caption=S3 Data Node example\cite{z6}]
        {
            "id" : "OutputData",
            "type" : "S3DataNode",
            "schedule" : { "ref" : "CopyPeriod" },
            "filePath" : "s3://myBucket/#{@scheduledStartTime}.csv"
        }
	\end{lstlisting}
	And the object example of \textbf{DynamoDBDataNode} is 
	\begin{lstlisting}[caption=DynamoDB Data Node example\cite{z7}]
        {
            "id" : "MyDynamoDBTable",
            "type" : "DynamoDBDataNode",
            "schedule" : { "ref" : "CopyPeriod" },
            "tableName" : "adEvents",
            "precondition" : { "ref" : "Ready" }
        }
	\end{lstlisting}
	In term of algorithm of activity, AWS Data pipeline provides several general activities to accommodate common scenarios. These activities include \textbf{CopyActivity}, \textbf{HiveActivity}, \textbf{PigActivity}, etc.\\ 
    
    \noindent Resource is used to make these activities work on data node. The AWS Data Pipeline supports two kinds of resources: EC2 and EMR. The concept of \textbf{Ec2Resource} is to use EC2 instance to perform the activity but \textbf{EmrCluster} is to use EMR cluster to perform the activity. 


\section{DynamoDB}
    \subsection{Overview}
    A step for modeling data structure and figure out the data type are necessary before uploading data into the DynamoDB. After that, the table will be created and there will be some implementation for the table such as updating, deleting.The Information viewpoint will be use to design the table. Algorithm viewpoint will be use to implement different operation in database.    
    \subsection{Information viewpoint}
    The Information viewpoint will be use to modeling data structure (design the tables).
	\subsection{ Design Concern}
	The major concern will be modeling the data structure of NoSQL database in a clear way. It is very important because it will be easy to create tables if the modeling part is clear. Another concern will be how to convert data into schema in NoSQL database.In relational database, the ER-diagram is usually used to convert data into schema. In the key-value type of NoSQL database, the ER-diagram is not suitable to represent the table.The DynamoDB is much different with the SQL type database such as MySQL.In DynamoDB,we treat each rows of sample data as an individual item and each columns will be the attributes for the item.When we load the data into table, we are actually load the items into the table in DynamoDB.The figure \ref{fig:1} shows the example for this.

\begin{figure}[H]
\includegraphics[width=15cm, height=12cm]{11.jpg}
\centering
\caption{\label{fig:1}sample csv data in DynamoDB table}
\end{figure}
 
	\noindent Based on the sample data, there will be three tables in the DynamoDB. The first table is called DataForcapstone1 which contains almost all the attributes except MAC address and ONID. Instead, there will be two linked tables which contains the MAC address and ONID. Each table will have unique keys and PINs, where the PINs are actually a hash of the ONID.By designing like this, the data analyst could give access control to different users for accessing the different tables. For example, a normal user might only have access to the main table, not the linked table.The figure \ref{fig:3} shows the process of this.

\begin{figure}[H]
\includegraphics[width=15cm, height=8cm]{7.jpg}
\centering
\caption{\label{fig:3}sample csv data in DynamoDB table}
\end{figure}
	
    \noindent The workflow of loading new data into DynamoDB will start at S3. At first, the New record will be stored into S3. After that, we will check whether the ONID or MAC is new. If it is not new, we will replace the Pin with ONID or MAC address in the new record and store back to the S3.If the ONID or MAC is new, we will generate Pins base on hash the ONID.Then, we will load the new PINs, ONID and MAC into DynamoDB Identity table. This table will not be accessible for the users. After that, we will load the anonymized record into DynamoDB. The figure \ref{fig:5} shows the process of this.
    
\begin{figure}[H]
\includegraphics[width=15cm, height=9cm]{8.jpg}
\centering
\caption{\label{fig:5} workflow for loading new data }
\end{figure}
 

    \subsection{Algorithm viewpoint}
    Algorithm viewpoint will be use to implement different operation such as create table, upload data, update table, delete table, list table, query table and scan table.
    \subsection{Design concerns}
    The concern for this part will be how to design the algorithm for operations. Fortunately, the Amazon provides basic APIs for the operations. So we can adapt it from the Amazon document API. 
    \subsection{Boto 3}
    Boto 3 is actually the AWS SDK for Python. It allows the Python developer to write program that make use of the services on AWS such as EC2. In our project, We could use it to do different implementations for the table.For instance, if we need to update some items in the table, we could use the APIs in AWS Python SDK to do the updating.For this project we use it to creating table, uploading data and other implementations.
	\subsection{APIs for operation}
 	 	\subsubsection{Create Tables}
    	We will use the Boto 3 for creating tables. The Amazon provides the Python API for creating tables.These API will help us to create the database schema that we need.The API will first create a DynamoDB class. After that it declares a table request which contains the table name, key of table, the definitions of attributes.\\
Here is the API\cite{w1} :
\begin{lstlisting}[language=python, caption=API for create tables]
import boto3
dynamodb = boto3.resource('dynamodb')
table = dynamodb.create_table( //create tables
    TableName=' ',
    KeySchema=[
        {
            'AttributeName': 'username',
            'KeyType': 'HASH'
        },
    ],
    AttributeDefinitions=[
        {
            'AttributeName': 'username',
            'AttributeType': 'S'
        },
    ],
\end{lstlisting}
    	\subsubsection{Create an new item}
    	We will use the Boto 3 for creating an new item in schema. The Amazon provides Python API for  creating an new item.We will use these API to creating an new item in schema. The API will first select the DynamoDB table.After that it declares a new item which contains the primary key and the attributes for creating.\\
Here is the API\cite{w1}:
\begin{lstlisting}[language=Python, caption=API for creating an new item]
table = dynamodb.Table('tablename')
table.put_item(
   	Item={
        'Username': ' ',
        'Attribute1': '',
        'Attribute2': '',
        'Attribute3': ,
        'Attribute4': ' ',
    }
}
\end{lstlisting}
    \subsubsection{Update Table}
    We will use the Boto 3 for updating table. The Amazon provides Python API for updating table.We will use these API to update the data in the table that we create.The API will first select the DynamoDB table. After that it select the item which needs to update and update the information.\\
Here is the API\cite{w1}:
\begin{lstlisting}[language=Python, caption=API for updating table]
table = dynamodb.Table('tablename')
table.update_item(
    Key={
        'username': '',
    },
    UpdateExpression='SET age = :val1',
    ExpressionAttributeValues={
        ':val1': 26
    }
)
\end{lstlisting}
    \subsubsection{Delete Item}
    We will use the Boto 3 for deleting item. The Amazon provides Python API for deleting items. We will use these API to deleting Item.The API will first select the DynamoDB table. After that, it deletes the item which needs to delete.\\
Here is the API\cite{w1}:
\begin{lstlisting}[language=Python, caption=API for delete data]
table = dynamodb.Table('tablename')
table.delete_item(
    Key={
        'username': 'keyname',
    }
}
\end{lstlisting}
    \subsubsection{Query table}
    We will use the Boto 3 for query the table. The Amazon provides Python API for querying the table.We will use these API to query the table that we create. The API will first select the DynamoDB table.After that, it will use the query to retrieving the data.\\
Here is the API\cite{w1}:
\begin{lstlisting}[language=Python, caption=API for query table]
table = dynamodb.Table('tablename')
response = table.query(
	KeyConditionExpression=Key('username').eq('')
)
items = response['Items']
print(items)
}
\end{lstlisting}
    \subsubsection{Scan Table}
    We will use the Boto 3 for scan the table.The Amazon provides  Python API for scan the table.We will use these API to scan the table in order to read the data which store in the table. The API will first select the DynamoDB table.After that, it will use the scan to scan the data.\\
    Here is the API for Scan Table\cite{w1}:
\begin{lstlisting}[language=Python, caption=API for scan table]
table = dynamodb.Table('tablename')
response = table.scan(
    FilterExpression=Attr('Attributes').lt( )
)
items = response['Items']
print(items)
\end{lstlisting}

 \section{Quicksight}
    \subsection{Overview}
    The data store in the DynamoDB will be used to create an analyze and a visual on Amazon visualization tool called Quicksight. In this part, Information viewpoint will be used.
    \subsection{Information viewpoint}
    The information viewpoint can be used to achieve visualization part. Basically, we will use the Quicksight as the visualization tool to represent the analyzing result.
    \subsection{Design concern} 
    The major concern for this part will be how to generate a visual on the Quicksight. Fortunately, the Amazon provides tutorial on the Quicksight document. Therefore, we can a follow the instruction step by step from the document.    
    \subsection{Step for creating a visual}
    Base on the Quicksight document, there will be several general step for creating a visual. The first step is to create a data source in your database. These data source are very important for Quicksight because it will require these data source for creating data set and do the analyzing. After that, the Quicksight needs to connect with the data source. In this step, if the database is on the Amazon web services the data source will be easily to connect. However, if the data source is not on the Amazon web services it requires port, database access information to connect the data source in database. The detail information is on the Amazon Quicksight document. The next step is to create the data set according to data source and do an analyzing. The last step is to create a visual according to the data set. There are different types of visual on the Quicksight and we can choose one to generate the suitable graph for the analyzing result.\cite{w2}For our project, the QuickSight will get the data source and create the data set from S3 or Local PC.\\
    The figure \ref{fig:2} shows the general process:
    \begin{figure}[H]
    \includegraphics[width=16cm, height=8cm]{6.jpg}
    \centering
    \caption{\label{fig:2}General process}
    \end{figure}



\newpage
        \thispagestyle{empty}
        \noindent\begin{tabular}{ll}
        \makebox[2.5in]{\hrulefill} & \makebox[2.5in]{\hrulefill}\\
        Client & Date\\[8ex]% adds space between the two sets of signatures
        \makebox[2.5in]{\hrulefill}\\
        Developer 1\\[8ex]
        \makebox[2.5in]{\hrulefill}\\
        Developer 2\\[8ex]
        \makebox[2.5in]{\hrulefill}\\
        Developer 3\\[8ex]
        \end{tabular}
\end{document}
