\begin{abstract}
	OSU campuses generate data constantly from multiples sources, including computer labs, wireless usage, student devices, and many others. This quantity of data, also known as big data, can effectively represent all kinds of behaviors of students for information technology. Currently, the data is very difficult to manage because it is collected from multiple sources and is impossible to analyze. Therefore, this project requires the use of various technologies for support. There are nine pieces of technologies required:1) Methods to measure performance metrics of database functionality. 2) Methods of database security. 3) Methods of user interaction with the system. 4) The framework and storage of processing unprocessed data 5) The ingestion and parsing for unprocessed data.  6) The operation for formatted and cleaned data in data storage 7) The storage way for dealing with processed data. 8) The programming language for achieving database functionality. 9) The visualization tool use to display the data. For each piece, we will provide the best three technology options. Our goal for this paper is to analyze each technology option and determine the optimal technology that we will implement.
\end{abstract}