	\section{Introduction}
		\subsection{Purpose}
		The purpose of this requirements document is to address specification for our product. We will clearly explain functionality, performance, attributes and design constraints of our product while remaining at a high-level to avoid technical details.The intended audience of this document is the client, our development group, and assessors of our work. Our group and client must share a mutual understanding of the project and all the details it entails, to confirm that our final product meets and matches all expectations. Assessors of our work may reference this document to compare it to our final product, again to ensure our final product matches requirements.
        
    	\subsection{Scope}
   	 	The name of this product is “Prototype Big Data Archive in a Public Cloud”, and will be used to to implement all operations about data from multiple sources and ensure data can be unified and organized onto a consistent cloud platform. The product can mine valuable information from data to integrate them because the data can reflect behaviors of students and staffs.\\
        
        \noindent A huge amount of data is generated when students and staffs use various information technologies such as printers and computers. The product will provide a database which can systematically deal with the data, so it contains several functions about operation for data such as ingest, store, manage and retrieve. On the other hand, the product is also able complete basic reporting and analysis. Therefore, these functions of product can help OSU Information Services staffs conveniently manage data, and they can directly and expediently understand behaviors of students and staffs according to analysis.

        \subsection{Definitions, acronyms, and abbreviations} 
        \begin{tabu} to \hsize {|X|X[2,l]|}
        \hline
        \textbf{Term} & \textbf{Definition}\\
        \hline
        User & The person who interact with our project \\
        \hline
        Big data & A holistic information management strategy that is used to integrate and management a large set of new types of data alongside traditional data. applications\cite{big data}\\
        \hline
        Cloud platform & The cloud platform is using the cloud computing technology.So in the cloud computing, Platform as a Service(PaaS) is used to integrate cloud-based computing environment such as running and development of applications. On the other hand, Infrastructure as a Service(IaaS) is used provide basic resources like operating systems and storage\cite{cplatform}\\
        \hline
        AWS & Amazon web service, a Platform as a service(PaaS) offered by Amazon \\
        \hline
        DB & Database\\
        \hline
        Nosql & non-relational database\\
        \hline
        \end{tabu}
        
        \subsection{References}
        
        \begin{thebibliography}{9}
    
        \bibitem{IEEE}\textit{IEEE Recommended Practice for Software Requirements Specifications}, The Institute of Electrical and Electronics Engineers, Inc., New York, NY, 1998.
        \bibitem{big data}"What is Big Data?", \textit{Oracle.com}, 2016. [Online]. Available: https://www.oracle.com/big-data/index.html. [Accessed: 11- Nov- 2016].
        \bibitem{cplatform}"What Is Platform as a Service (PaaS) in Cloud Computing?", \textit{dummies}, 2016. [Online]. Available: http://www.dummies.com/programming/cloud-computing/hybrid-cloud/what-is-platform-as-a-service-paas-in-cloud-computing/. [Accessed: 11- Nov- 2016].
        \bibitem{AWS Rate}"AWS Rates Highest on Cloud Reliability", \textit{EnterpriseTech}, 2016. [Online]. Available: http://www.enterprisetech.com/2015/01/06/aws-rates-highest-cloud-reliability/. [Accessed: 08- Nov- 2016].
        \bibitem{Cloud}"Advantages and Disadvantages of Cloud Computing", \textit{LevelCloud}, 2016. [Online]. Available: http://www.levelcloud.net/why-levelcloud/cloud-education-center/advantages-and-disadvantages-of-cloud-computing/. [Accessed: 08- Nov- 2016].
        
        \end{thebibliography}

        \subsection{Overview}
        Following this introduction is a section describing the product as a whole. This includes perspective, function, user characteristics, constraints, and assumptions/dependencies. The purpose of the section is to provide in-depth details on requirements of the product. Finally, this document is concluded with a section on specific requirements.
    