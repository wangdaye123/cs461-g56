\subsection{Fall Term}
\subsubsection{Week 3}

\textbf{Zhi Jiang}\\
\noindent\textit{Progress since last week:}\\
Last week we completed the first assignment problem statement. The assignment effectively made me understand the entire condition of this project. This assignment helped me quickly collect all kinds of information about this project. On the other hand, I still learned two kinds of markup languages, and which are LaTex and Markdown. These markup languages will be useful tool for I working this project.\\

\noindent\textit{Any problems you encountered:}\\
\noindent The most important problem we encountered is database. According to client's comments, we need to evaluate some ways to implement NoSQL database, but actually we are not familiar with database. I spent a lot of time on searching materials about NoSQL database when I modified our problem statement. Another problem is we should have more effective communication with the client. The Client told us it would be a good idea to let him know of deadlines, so we should ensure information symmetry for client.\\

\noindent\textit{Plans for the coming week:}\\
\noindent We will keep going to focus on database in next week. I will try to send email to my instructor in CS 340 and ask him some questions about NoSQL and way of implementation of NoSQL. We will find more details in this project, in the other words, we will do a complete analysis for it. \\

\noindent\textbf{Isaac T Chan}\\
\noindent\textit{Progress since last week:}\\
\noindent Last week we wrote the problem statement and sent it and received it from the client. There was only one revision necessary which was good, meaning we understand the problem that he presented fairly accurately.\\

\noindent\textit{Any problems you encountered:}\\
\noindent We somehow got a bit behind on deadlines and sent it to the client too late. However he was very good about reading and signing it so we were lucky in that respect. We will be much more prompt in the future.\\

\noindent \textit{Plans for the coming week:}\\
\noindent I don't know the difference between relational and non-relational databases and need to learn. We will probably want to fill in the gaps in our personal knowledge and start thinking about a plan to implement.\\

\noindent\textbf{Zhaoheng Wang}\\
\noindent\textit{The process we finish last week:}\\
\noindent We finish our problem statement part. So I figure out the general process of project about what should we do and how to evaluate whether our project meets the requirements.However,it is not specific enough.\\

\noindent\textit{Any problems you encountered:}\\
Our project requires to use the amazon web service as an platform. However, I never use it before and I am not familiar with this platform. Besides, I am still confused about the concept of NoSQL database.\\

\noindent\textit{Plans for the coming week:}\\
Firstly, I would like to search more information about this platform about how to use the AWS. Secondly, our project will focus on collecting data, organizing and analyzing data and implement the database to store the data. Therefore, I should figure out how to implement a NoSQL database on AWS.Furthermore,I might search some videos about what is the difference between NoSQL and SQL.\\

\subsubsection{Week 4}

\textbf{Zhi Jiang}\\
\noindent\textit{Progress since last week:}\\
In the last week, we have done three things. First one is we fixed some minor mistakes in problem statement according to instructor's feedback. Secondly, we sent our modified document to client and made an appointment with him in next week. Third thing is we met with our TA and talked somethings about requirements documents. As for me, I reviewed some commands of Git because I think I am not familiar to this tool.\\

\noindent\textit{Any problems encountered:}\\
The most important problem we encountered should be we still have a lot of problems about our project, in the other word, we need more details such as implementation of database and types of sample data.\\ 

\noindent\textit{Plans for the coming week:}\\
We will focus on requirements document in next week.  As I mentioned above, we made an appointment with our client in next week because we need more specific information about project. After meeting with client, we will start to do requirements document. We will read related material and understand how to write correctly requirements. For instance, we need to know differences between functional requirements and non-functional requirements.\\

\noindent\textbf{Isaac T Chan}\\
\noindent\textit{Progress since last week:}\\
There was not a lot of progress this last week. Because of the extended revision deadline for the problem statement, we were on hold until we received feedback for it. When we did receive feedback, it required very minimal corrections before we sent it again to our client. We also met with our TA for the first time, made plans for meeting with our client, and have a plan for the requirements document, due next week.\\

\noindent\textit{Any problems encountered:}\\
There were no problems we encountered. I did worry a little about the condition of our problem statement because we had only a short time to fix problems and get it approved and signed by our client, but was relieved to see that the feedback was positive.\\

\noindent\textit{Plans for the coming week:}\\
This next week will be busy. The requirements document seems to need a lot of work. I am not that experienced in LaTeX - some time will be spent on the IEEE formatting. Additionally, we hope to meet with our client as early in the week as possible, so we can clarify some details and be ready to write an excellent requirements document.\\ 

\noindent\textbf{Zhaoheng Wang}\\
\noindent\textit{Progress since last week:}\\
In the last week, we revise our problem statement base on the feedback. After that, we send it to our client to check whether fits the requirement about the project. Besides, we meet our TA on Friday. we go though the general process about future meeting and we also ask several question about requirement document such as how detail should our requirement document should be. Furthermore, I do some research about NOSQL database which required by our project. it seems the biggest difference between SQL and NOSQL is SQL database uses the table and the NOSQL uses the document. The NOSQL is more focus on availability and performance.\\     

\noindent\textit{Any problems encountered:}\\
For the requirements document, the lecture shows five general parts about what we need to write such as functionality, security, response time. However, i could not give a clear answer for several parts such as security. For example, we are going to use the database on AWS. Usually, there will be some disaster recovery for the database and we might figure out how to do it.\\   

\noindent\textit{Plans for the coming week:}\\
we are going to have a meeting with our client and figure out how to organize our requirements document. Additionally,I am not familiar with latex style so i would like do some research on it. Furthermore, if I have enough time after finishing the requirements document and the research about latex, I will focus on searching the information about how to use Amazon Web Services.

\subsubsection{Week 5}

\textbf{Zhi Jiang}\\
\noindent\textit{Progress since last week:}\\
In this week, we finished the final draft of problem statement and started to write software requirement specification. This document needs more details about requirements, thus we met our TA again and then we gained many information. In process of writing document, we feel benefits of requirement specification, because we need to completely understand what we do in project, and provide some specific plans to each part, otherwise we cannot complete this document very well.\\ 

\noindent\textit{Any problems encountered:}\\
The third part of document "the specific requirements" is main problem we encountered, because we still have many questions about this part such as interfaces and functional requirements. In fact, according to client's introduction, our project is not very complex and the main part is database, so we will consider which section need to be ignored in this document.\\

\noindent\textit{Plans for the coming week:}\\
The document is due next week, so the primary task for us is complete this document bases on feedback of TA and client. Probably we need to search more material about database, and find useful information for specific requirements.\\

\noindent\textbf{Isaac T Chan}\\
\noindent\textit{Progress since last week:}\\
This week we completed our rough draft of the requirements document. We realized early that there were some requirement details that we didn't know; we met with our client early on to define them. Then we wrote the rough draft and gave it to our client to check the content.\\

\noindent\textit{Any problems encountered:}\\
We still need to put the document into IEEE format. Also, there was some confusion regarding the "specific requirements" section of the document. It seems like we've covered all the requirements already in the previous section. This is likely an organizational issue and we will need to figure it out.\\

\noindent\textit{Plans for the coming week:}\\
Luckily we have another week to turn in the final draft. We will need to put it into IEEE format. Regarding the missing section, we may need to meet with an instructor or TA to reorganize our document. Finally we need to finish it before Friday so we will have enough time to verify requirements one last time with our client, get it signed, and turn it in.\\

\noindent\textbf{Zhaoheng Wang}\\
\noindent\textit{Progress since last week:}\\
This week we have finished our rough draft of requirements document and we send it to the client in order to get some feedback. After that, we revised it base on the feedback form client.Then,we send it to our TA in order to get more feedback.Besides, I have do some research about the idea between cloud platform ,PaaS and IaaS.The cloud platform is using the cloud computing technology. The Platform as a service (PaaS) offered by cloud platform will provide ​a development environment for the product including OS, compiling and execution environment of programming language.The Infrastructure as a service(​IaaS)​ will ​abstract the details of infrastructure​ such as ​physical computing resources​, security.Thus, the user don’t need to worry about managing ​cloud infrastructure\\    

\noindent\textit{Any problems encountered:}\\
we have some question about the specific requirement part in requirements document. Because, we are not clearly figure out all the specific functions we need to use yet.\\

\noindent\textit{Plans for the coming week:}\\
Next week, we will do some revision after getting the feedback from TA. Then, send it to our client to check whether fit his requirement. Besides, I would like to do some research about some analyze tool such as tableau because our project need to have some basic reporting and analyzing functionality.    

\subsubsection{Week 6}

\textbf{Zhi Jiang}\\
\noindent\textit{Progress since last week:}\\
In this week, we talked grade of problem statement with Kirsten, and she gave us some useful suggestions about writing structure. We will do appropriate adjustment for our problem statement through her comments. Secondly, we finished the requirements document but our TA thinks it is not good enough. He thinks we need more stuffs in "Specific requirements" section. I still learned some knowledge about technology when I was doing the requirements document. I download the document of Dynamo DB for developer and read it. I find that it has provided some APIs of basic operations of NoSQL Database. These APIs will make our tasks more convenient.\\

\noindent\textit{Any problems encountered:}\\
 The problem we need to solve are mistakes in the requirements document. We have to change our format of Latex and we also need to provide more details for functional requirements according to TA feedback. So it means we need to do more research about NoSQL database and completely understand its functions.\\

\noindent\textit{Plans for the coming week:}\\
We will modify our requirements document as soon as possible and return it to client and TA. We still need to learn more thing about NoSQL.\\

\noindent\textbf{Isaac T Chan}\\
\noindent\textit{Progress since last week:}\\
This week we received our graded problem statement back and finished up the final draft of our requirements document. We made an appointment with Kirsten to talk about our problem statement grade and the "specific requirements" section of our requirements document. We plan to reorganize and resubmit our problem statement, because the content was good but the structure was poorly designed.\\

\noindent\textit{Any problems encountered:}\\
Our requirements document is likely not in "final draft" shape. We received feedback from our client, who thinks it is good, and also our TA, who thinks it is not so good. Due to the delay of feedback we may need to get an extension and turn it in at a later date.\\

\noindent\textit{Plans for the coming week:}\\
 Probably will need to finish up the requirements document and restructure the problem statement. Plus any new assignments.\\

\noindent\textbf{Zhaoheng Wang}\\
\noindent\textit{Progress since last week:}\\
In this week, we make an appointment with Kirsten about our problem statement and get useful feedback from her. we find that our problem statement need to organize the information more clearly for the readers. Besides, we finish our requirements document and get some feedback from our TA. Furthermore, i do some research about the whole process again and i have a more clearly idea about how this whole process goes.\\

\noindent\textit{Any problems encountered:}\\
Our requirements document still have some problems: 1. The syntax for latex need to change 2. The content for specific requirements part should be more specific 3. The format for specific requirement part need to change 4. Adding a schedule for the requirements\\

\noindent\textit{Plans for the coming week:}\\
We need to revise our Requirements document and problem statement again. Besides, I would like do more research about loading data from different source and the algorithm for searching specific data in a file. i believe this will help a lot for the detail design part.\\

\subsubsection{Week 7}

\textbf{Zhi Jiang}\\
\noindent\textit{Progress since last week:}\\
In this week, the first thing is that we completed the final draft of requirements document. In this process, we add two more sections "Hardware interface" and "Software interface". We added more details for functional requirements. Eventually we created our schedule bases on functional requirements. On the other hand, I started to prepare for writing technology review. I have found many materials about each piece I will deal with so they will be useful for me to complete this assignment.\\

\noindent\textit{Any problems encountered:}\\
Actually, when I read those materials about my pieces, I was frustrated, because there are many concepts I don't know. It is hard to understand relationship between some technologies, so I believe that this is the biggest problem I encountered.\\

\noindent\textit{Plans for the coming week:}\\
In next week, we will start to do technology review, so it means we must understand completely all pieces in our product, so probably we need to find more materials or ask corresponding questions to TA or instructor.\\

\noindent\textbf{Isaac T Chan}\\
\noindent\textit{Progress since last week:}\\
We completed the final draft of our requirements document. We added functional requirements, software system attributes, and added more details for performance metrics. We also added a rough schedule based on our functional requirements. There is a rough outline of our technology review.\\

\noindent\textit{Any problems encountered:}\\
In researching for the technology review, I found that with the information we have right now about the data and the system, it is difficult to envision the final product. I think we need to get a sample of data to work with, or all of our design and reviews are just speculation and our final product could deviate far from the preliminary design.\\

\noindent\textit{Plans for the coming week:}\\
We plan to finish our technology review and start the design document. We will need more details on the design document and may need intermediate meetings with our TA to make sure we're doing it correctly.\\

\noindent\textbf{Zhaoheng Wang}\\
\noindent\textit{Progress since last week:}\\
In this week, we revise our Requirements document again base on our TA's feedback. We first change the functional requirement part and make this part more detail. Besides, we add performance requirement part and system attribute part which let reader understand our system more clearly. Furthermore, we make a schedule in gantt chart which give a clearly schedule for our project.\\

\noindent\textit{Any problems encountered:}\\
For the technology review, i am not familiar with how to make the data suitable before visualization so i need to do more research about this part. Besides, i also need to figure out what kind of tools we might use for visualization.\\

\noindent\textit{Plans for the coming week:}\\
we might start our detail design next week. Therefore, it is better to ask an appointment with our client and data analyzer. we need to check our thought with our client and make sure it fits our client's requirement. For example, we need to ask for some sample data from our client so we could figure out the detail process for loading data. Besides, we also need to check our thought with our user(data analyzer) and figure out what types of data we need to create for loading into the analyze tool.\\

\subsubsection{Week 8}

\textbf{Zhi Jiang}\\
\noindent\textit{Progress since last week:}\\
We completed technology review and submitted it on Monday. And then we started to do the design document. In this document, we need to focus on how to implement technologies we discussed in the last document. When we met our TA today, we asked some question about our technologies such as what's the difference between data storage and database. TA given us useful suggestions and explanation. And he also provided some basic ideas of big data,so these are good reminders for us to consider the whole product.\\

\noindent\textit{Any problems encountered:}\\
The problem is also how we implement these technologies. We need to do more research base on suggestions of TA because we are not familiar with some technologies like the filesystem.\\

\noindent\textit{Plans for the coming week:}\\
 As for next week, probably we will meet with our client and talk something about sample data. According to TA's suggestion, we should get sample data as soon as possible because we need to design our product depends on characteristics of these data. On the other hand, we will make a video for our product.\\

\noindent\textbf{Isaac T Chan}\\
\noindent\textit{Progress since last week:}\\
This week we finished our technology review. Not a lot was discovered because we had previously researched technologies we would use, and our research affirmed those technologies. Additionally, I think we found that, because our client requires us to use AWS, AWS actually offers a lot of the technologies we need. This is very helpful because it reduces the complexity of our implementation.\\

\noindent\textit{Any problems encountered:}\\
Our problems remain the same as last week. We still need a sample of the data to determine specific implementation details and how to design the database.\\

\noindent\textit{Plans for the coming week:}\\
I plan on restructuring our problem statement next week and resubmitting it. I also plan on completing a portion of the design document. I think that we have fallen behind schedule slightly and I think the progress report presentation will take more time than we think. As it's already the end of week 8, we have a lot of work ahead.\\

\noindent\textbf{Zhaoheng Wang}\\
\noindent\textit{Progress since last week:}\\
 on this week, i finish our technology plan which is going to figure out the optimal options for achieving the requirements. This document is a necessary step for our design documents and it is very important. In this document, i choose three pieces of technologies: the first one is to figure out which is the optimal option of the storage way for dealing with processed data. In this part, i compare the other storage way with database. After that,I compare the sql and nosql database. Finally, I compare three types of nosql databases.because of the research, I benefit a lot and get more clearly understanding for the features of database than before.\\
 
\noindent\textit{Any problems encountered:}\\
 i am going to start our design document however there are several things we have to check with our client: 1 i need to get some sample data first because our first step is to parse the file. 2 i need to figure out the information we need to store. 3 For the visualization part, i need to check with data analyzer because he is our user.\\
 
\noindent\textit{Plans for the coming week:}\\
we are planning to have an appointment with our client next week and figure out the questions above.Besides, there are many research for our project. it is very important to get a clear scope of our project before we start to design it. furthermore, we should start to think about the progress report on next week.
\\\textbf{}

\subsubsection{Week 9}

\textbf{Zhi Jiang}\\
\noindent\textit{Progress since last week:}\\
In this week, we met with our client on Monday. The main thing we talked is about sample data. We are doing the design plan for entire product, so sample data can let us understand how to design each part. Another important thing was we completed the revision of problem statement and submitted to Winters. As for myself, we found more information about solution I chose for my parts. I read some developer guide provided by AWS and known some basic idea for building Hadoop framework.\\

\noindent\textit{Any problems encountered:}\\
The problem we encountered is we have not got sample data from our client, so it will affect our plan, because sample data is very important for us and all of tasks we will do must depend on sample data.\\

\noindent\textit{Plans for the coming week:}\\
The primary is to get sample data for comping week and then complete design plan as soon as possible. I got grade for my technology review, and I will probably talk feedback with Winters in next week.\\

\noindent\textbf{Isaac T Chan}\\
\noindent\textit{Progress since last week:}\\
We met with our client early in the week and requested a set of sample data. This is very important for the design document because our database structure depends on how the sample data is set up. There are many components that are reliant on the sample data, such as whether or not we need to cleanse, parse, or otherwise process the data before it can be stored in the database. We also restructured our problem statement and sent it to Dr. Winters.\\

\noindent\textit{Any problems encountered:}\\
We are waiting for the set of sample data from our client. Due to the break, we don't know when to expect it but as soon as it does we can inspect it and base some design decisions off of it.\\

\noindent\textit{Plans for the coming week:}\\
Hopefully we receive the set of data as soon as possible. Then we can organize and collaborate on the design document. Additionally at some point we should write the progress report and begin the presentation.\\

\noindent\textbf{Zhaoheng Wang}\\
\noindent\textit{Progress since last week:}\\
This week we make an appointment with our client and ask for a sample data. The sample data is very important for design document because if we get the sample data then we will figure out the data type we should parse for our project. Besides, the sample data will give a good example about the information we should store into database. I also watch some video about how dynamodb works however it is not very helpful.\\

\noindent\textit{Any problems encountered:}\\
The problem is i still confuse about the way that dynamodb store and there are several questions i need to figure out: 1. Does dynamodb requires er diagram? 2. How quicksight works on aws? 3. How to achieve the functionality for database? another thing is we don't get our sample and we could not start our design part.\\

\noindent\textit{Plans for the coming week:}\\
Once we get the sample data, it is time to start our design document. Furthermore, we might also start to think about the process report and think about how to organize it.

\subsubsection{Week 10}
\textbf{Zhi Jiang}\\
\noindent\textit{Progress since last week:}\\
In this week, we gained the sample data from our client and then completed the design document. According to this document, we understand more details about our technologies. We also made a new timeline in this document because there are some changes which are different from previous documents.\\

\noindent\textit{Any problems encountered:}\\
The most difficult problem is also about our technologies. We still need to spend more time to learn how to use these technologies. I think we can start to use sample data to attempt some basic functions for tools on AWS.\\

\noindent\textit{Plans for the coming week:}\\
We plan to complete the last assignment on Sunday. And we still need to submit the hard copy with signature as soon as possible.\\ 

\noindent\textbf{Isaac T Chan}\\
\noindent\textit{Progress since last week:}\\
This week we completed our design document. In the beginning of the week we received a sample of data from our client. This data was in a CSV format, which should be simple to load into the database. During our work with the design document, we learned much more about the technologies we will implement. After seeing the format of the data and what types of fields it contains, it made me more confident in my technology choices because I had no conflicts with the data.\\

\noindent\textit{Any problems encountered:}\\
I think that we are still in hypothetical terms when designing our project. The problems we will encounter will be small nuances in the technologies we choose. As of now, I don't think any of us have issues with our technologies or the sample data, but the issues we may have are very difficult to identify without actually implementing the project. Now that we have sample data, we can probably begin to work with it on AWS and experiment, but we will need credentials given to us by our client.\\

\noindent\textit{Plans for the coming week:}\\
We plan to complete the progress report on Sunday, 12/4. Also we submitted an unsigned version of the design document and will need to submit the signed version as soon as it is returned to us.\\

\noindent\textbf{Zhaoheng Wang}\\
\noindent\textit{Progress since last week:}\\
This week, we receive our sample data and finish our preliminary design document.In this document, we introduce our technology based on various such as context viewpoint,algorithm viewpoint and information viewpoint.This document gives us more clear visual for our project.Besides,I have done some personal research about the process for designing NoSQL table.Usually, the relational database table will design by ER-diagram but the ER-diagram is not suitable for represent NoSQL database table.However, the ER-diagram can convert to the NoSQL database table model.Therefore, we can create the ER-diagram first and convert it into the NoSQL database table model.\\

\noindent\textit{Any problems encountered:}\\
The sample data provided by data analyzer is in csv file type which is already readable by using excel.In this file, it contains the data such as userid,time,terminal type, OS and others.Therefore, there will be two things I need to check with data analyzer.The first thing is whether the future data is the same type with the sample data. The second thing is whether the sample data contains all the information we need to store.\\

\noindent\textit{Plans for the coming week:}\\
In the next week, we will finish our process document for this term. Besides, I might start to think the specific step about how to implement this project.\\